\documentclass[aps,twocolumn,secnumarabic,nobalancelastpage,amsmath,amssymb,
nofootinbib,superscriptaddress]{revtex4-1}


\usepackage{graphics}       % standard graphics specifications
\usepackage{graphicx}       % alternative graphics specifications
\usepackage{longtable}      % helps with long table options
\usepackage{url}            % for on-line citations
\usepackage{bm}             % special 'bold-math' package
\usepackage[ngerman]{babel} % deutsche Siblentrennung
\usepackage[utf8]{inputenc} % Umlaute

\def\andname{\hspace*{-0.5em},} % definiert die Trennung zwischen 2 Autoren neu

% Titelseite
\begin{document}
\title{Quantisierter Leitwert von Punktkontakten}
\author         {Ch. Egerland}
\email[Email: ]{egerlanc@physik.hu-berlin.de}
\author         {M. Pfeifer}
\email[Email: ]{max.pfeifer@physik.hu-berlin.de}
\affiliation    {Humboldt-Universität zu Berlin, Institut für Physik}
\date[Versuchsdatum: ]{22.06.2017}

%%%%%%%%%%%%%%%%%%%%%%%%%%%%%%%%%%%%%%%%%%%%%%%%%%%%%%%%%%%%%%%%%%%%%%%%%%%%%%%%
\begin{abstract}
Lorem ipsum dolor sit amet, consectetur adipiscing elit. Nam id facilisis ligula,
a ultrices nibh. Nullam suscipit tellus nec mauris fermentum, ornare luctus neque
tincidunt. Aenean commodo tincidunt varius. Phasellus faucibus metus non erat
consectetur bibendum. Duis et luctus risus, at egestas justo. Nunc eleifend lacus
ac laoreet scelerisque. Aenean cursus dignissim magna in ultrices. In eget nisl
quis nisi.
\end{abstract}


\maketitle



%%%%%%%%%%%%%%%%%%%%%%%%%%%%%%%%%%%%%%%%%%%%%%%%%%%%%%%%%%%%%%%%%%%%%%%%%%%%%%%%

\section{Theorie}

Der Leitwert ist das Inverse des Widerstandes und ist ein Maß dafür, wie gut ein
Material Strom leitet. Die Quantisierung des Leitwertes kann wie folgt erklärt
werden: Wir modellieren den Quantenpunktkontakt als ein effektives Kastenpotential
mit Breite $d_x$ und Dicke $d_y$. Es bilden sich senkrecht zur Bewegungsrichtung
stehende Elektronenwellen mit der Energie

  \begin{equation}
    E_{lm} = \frac{\pi^2 \hbar^2}{2 m_e} \left(\frac{l^2}{d_x^2}+\frac{m^2}{d_y^2} \right)
  \end{equation}

In Bewegungsrichtung haben wir unter den Voraussetzungen ein kontinuierliches
Energiespektrum mit $E_z = \hbar^2 k_z^2 / 2m_e$. Die Gesamtenergie der elektronischen
Zustände ist dann: $E_{ges} = E_{lm} + E_z$.
Die Zustandsdichte in einem eindimensionalen System ist gegeben durch:

  \begin{equation}
    D(E) dE = \frac{1}{\pi \hbar} \sqrt{\frac{m}{2 E}}dE
  \end{equation}

Bei Anlegen einer kleinen Spannung $dV$ folgt ein kleiner Strom $dI = e v dn$,
wobei $dn = D(E)dE$. Somit ist (mit $v = \sqrt{2E/m}$):

  \begin{equation}
    G = \frac{dI}{dV} = \frac{evD(E)dE}{dV} = \frac{e^2vD(E)dV}{dV} = \frac{2e^2}{h}
  \end{equation}






%%%%%%%%%%%%%%%%%%%%%%%%%%%%%%%%%%%%%%%%%%%%%%%%%%%%%%%%%%%%%%%%%%%%%%%%%%%%%%%%
\section{Experiment}

Text zum Experiment mit vielleicht einer Grafik:

\begin{figure}[h]

\caption{ein Bild vom Versuchsaufbau, Aus: \cite{melissinos1966,melissinos2003}.}
\label{fig:samplefig}
\end{figure}

Lorem ipsum dolor sit amet, consectetur adipiscing elit. Nam id facilisis ligula,
a ultrices nibh. Nullam suscipit tellus nec mauris fermentum, ornare luctus neque
tincidunt. Aenean commodo tincidunt varius. Phasellus faucibus metus non erat
consectetur bibendum. Duis et luctus risus, at egestas justo. Nunc eleifend lacus
ac laoreet scelerisque. Aenean cursus dignissim magna in ultrices. In eget nisl
quis nisi.



%%%%%%%%%%%%%%%%%%%%%%%%%%%%%%%%%%%%%%%%%%%%%%%%%%%%%%%%%%%%%%%%%%%%%%%%%%%%%%%%
\section{Daten und Analyse}

Lorem ipsum dolor sit amet, consectetur adipiscing elit. Nam id facilisis ligula,
a ultrices nibh. Nullam suscipit tellus nec mauris fermentum, ornare luctus neque
tincidunt. Aenean commodo tincidunt varius. Phasellus faucibus metus non erat
consectetur bibendum. Duis et luctus risus, at egestas justo. Nunc eleifend lacus
ac laoreet scelerisque. Aenean cursus dignissim magna in ultrices. In eget nisl
quis nisi. Tabelle \ref{tab:table1}:


\begin{table}[h]
\caption{\label{tab:table1}Eine Tabelle mit Fußnoten}
\begin{ruledtabular}
\begin{tabular}{cccccccc}
 &$r_c$ (\AA)&$r_0$ (\AA)&$\kappa r_0$&
 &$r_c$ (\AA) &$r_0$ (\AA)&$\kappa r_0$\\
\hline
Cu& 0.800 & 14.10 & 2.550 &Sn\footnotemark[1] & 0.680 & 1.870 & 3.700 \\
Ag& 0.990 & 15.90 & 2.710 &Pb\footnotemark[1] & 0.450 & 1.930 & 3.760 \\
Tl& 0.480 & 18.90 & 3.550 & & & & \\
\end{tabular}
\end{ruledtabular}
\footnotetext[1]{Entnommen aus Ref.~\cite{bevington2003}.}
\end{table}




%%%%%%%%%%%%%%%%%%%%%%%%%%%%%%%%%%%%%%%%%%%%%%%%%%%%%%%%%%%%%%%%%%%%%%%%%%%%%%%%
\section{Schlussfolgerung}

Schlussoflgerung, sollten wir mal was von nem Buch oder so entnehmen nutzen wir:


\begin{quote}
  Ein Zitat mit Referenz auf das Buch\cite{melissinos1966}
\end{quote}

Lorem ipsum dolor sit amet, consectetur adipiscing elit. Nam id facilisis ligula,
a ultrices nibh. Nullam suscipit tellus nec mauris fermentum, ornare luctus neque
tincidunt. Aenean commodo tincidunt varius. Phasellus faucibus metus non erat
consectetur bibendum. Duis et luctus risus, at egestas justo. Nunc eleifend lacus
ac laoreet scelerisque. Aenean cursus dignissim magna in ultrices. In eget nisl
quis nisi.


%%%%%%%%%%%%%%%%%%%%%%%%%%%%%%%%%%%%%%%%%%%%%%%%%%%%%%%%%%%%%%%%%%%%%%%%%%%%%%%%
\bibliography{sample-paper}
\bibliographystyle{prsty}
\begin{thebibliography}{99}
\bibitem{wees88}B. J. van Wees et al. : Quantized Conductance of Point Contacts in a Two-Dimensional Electron Gas, Physical Review Letters Vol. 60 Nr.9 (1988)
\bibitem{wharam88}D. A. Wharam et al. : One-dimensional transport and the quantisation of the ballistic resistance Verlag, Phys. C: Solid State Phys. 21  (1988)
\bibitem{apetrii04}Gabriela Apetrii : Quantum point contacts with one and two vertical modes fabricated with an atomic force microscope, Dissertation (2004)
\bibitem{vanhouten96}Henk van Houten et al. :  Quantum Point Contacts, Physics Today July S. 22–27, (1996)
\bibitem{knop07}Michael H. Knop :  Ballistische Gleichrichtung in asymmetrischen elektronischen Wellenleiterkreuzen, Dissertation (2007)
\bibitem{fischer02}S. F. Fischer et al. :  Control of the confining potential in ballistic constrictions using a persistent charging effect, Applied Physics Letters Vol. 81 Nr.15 (2002)
\bibitem{apetrii02}G. Apetrii et al. :  Influence of processing parameters on the transport properties of quantum point contacts fabricated with an atomic force microscope, Institute of Physics Publishing Semicond. Sci. Technol. 17 S. 735–739 (2002)
\end{thebibliography}


%%%%%%%%%%%%%%%%%%%%%%%%%%%%%%%%%%%%%%%%%%%%%%%%%%%%%%%%%%%%%%%%%%%%%%%%%%%%%%%%
\clearpage
\appendix

\section{Sonstiges}
Hier sehen wir einen Beispiel Anhang und so könnte man Code in Latex einbinden:
\begin{verbatim}
> mkdir ~/8.13
> mkdir ~/8.13/papers
> mkdir ~/8.13/papers/template
> cd ~/8.13/papers/template
\end{verbatim}


%%%%%%%%%%%%%%%%%%%%%%%%%%%%%%%%%%%%%%%%%%%%%%%%%%%%%%%%%%%%%%%%%%%%%%%%%%%%%%%%


\end{document}
