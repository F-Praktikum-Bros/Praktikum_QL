% Dokumentenklasse
\documentclass{article}

% deutsche Silbentrennung
\usepackage[ngerman]{babel}

% Tabellenpaket
\usepackage{tabularx}

% Umlaute
\usepackage[utf8]{inputenc}

% Fließobjekte
\usepackage{float}

% Mathepaket
\usepackage{amsmath}

% Bilder
\usepackage{graphicx}

% Für schräge Bruchstriche
\usepackage{nicefrac}

% Für Bilderformatierung
\usepackage{float}

% PDFs anhängen
\usepackage{pdfpages}

% Hyperref
\usepackage{hyperref}

% Literaturverzeichnis
\usepackage{cite}

% Latexbilder
\usepackage{tikz}

\bibliographystyle{unsrt} % <- Choose a style. Your document class may do this for you.
\bibliography{my-biblio}


\begin{document}

\begin{titlepage}
\centering
Protokoll zum Fortgeschrittenenpraktikumsversuch\\[2cm]\Huge
\vfill
\Huge
Zeitkorrelierte Einzelphotonenzählung\\[2cm]\Large
WS 2016/2017 \\
\vfill
\normalsize
\begin{tabular}{lcr}
Verfasser 1:   &   & Christoph Egerland       \\
Verfasser 2:   &   & Max Pfeifer              \\
Versuchsdatum: &   & 6.12.2016                \\
Versuchsplatz: &   & NEW15 2'106              \\
Betreuer:      &   & Dr. Steffen Hackbarth    \\
\end{tabular}
\end{titlepage}

\begin{center}
  \textbf{Abstract}
\end{center}
Hier steht der Abstract

\tableofcontents
\newpage


\section{Materialien und Methoden}
\subsection{Versuchsaufbau}
\subsection{Versuchsmethode}

\section{Auswertung und Diskussion}

\section{Schlussfolgerungen}

\section{Anhang}



\section{Literatur}

\begingroup
\renewcommand{\section}[2]{}
\begin{thebibliography}{20}
        \bibitem{Hackbarth} Dr. Steffen Hackbarth: \textit{Versuchsskript: Zeitkorrelierte Einzelphotonenmessung}
        \bibitem{Roeder} Röder et al.: \textit{Photophysical properties of pheophorbide a in solution and in model membrane systems}
\end{thebibliography}
\endgroup

\end{document}
